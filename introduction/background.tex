%!TEX root = ../thesis.tex

\section{背景}
二足歩行ロボットの歩行技術はVukovratvicらのZMP\cite{VUKOBRATOVIC19721}から始まり,今日まで発展を続け,様々な手法が発表されているが、その中で共通の歩くために必要な1つの条件として,転倒しない,というものがある.一般に,二足歩行ロボットに歩行動作を行わせる場合はその,転倒しない,という条件を常に満たした状態を保つ事が必要とされる.そのため,二足歩行ロボットの歩行パターン生成は,この条件を達成するために地面との接触力など様々な制約を考慮して行われることになる.しかし,歩行パターン生成を行う中で,これらの制約条件を陽に表現する事は難しく,LQRなどの最適制御理論などを使う場合は陰に制約を埋め込むなどの工夫がされてきた.これに対して,明示的に制約を表現し,それらを考慮して制御を行う事が可能な手法である,モデル予測制御(以下,MPC\cite{MPC}は上記の条件と相性が良く,近年,MPCを利用した歩行パターン生成手法が多く発表されている.
しかし,発表数の多さにも関わらず,MPCを利用した歩行パターン生成手法の公開された実装は非常に少ない.現在,インターネット上で公開されている実装は3件\cite{GITHUB1}\cite{GITHUB2}\cite{GITHUB3}に留まっている.これにより,MPCを利用した歩行パターン生成をユーザとして利用したい場合,また新たに学びたいといった場合には個別に都度ノウハウや実装手法を自らの手で一から構築する必要があるという問題がある.

\section{目的}
本研究では、MPCによる歩行パターン生成手法の公開された実装が少なく、新規利用者が個別に都度ノウハウや実装手法を自ら構築する必要があるという問題を解決するために、論文などで紹介されているMPCを利用した二足歩行パターン生成手法を実装し、ソースコードと実装時に得た知見などをGithub\cite{MYGITHUB}で公開する。

\section{論文構成}
本論文の構成は全5章からなる。2章では、モデル予測制御を利用した二足歩行ロボットの歩行パターン生成手法について、二足歩行ロボットの歩行に関する基本的な概念から今回実装した手法までを記載した。3章では公開するソフトウェアの実装手法について、知見などと共に記載した。4章では本研究で実装したソフトウェアで歩行パターンを生成しその結果について記載した。5章では本研究のまとめと今後の展望について記載した。

% \begin{figure}[hbtp]
%   \centering
%  \includegraphics[keepaspectratio, scale=0.8]
%       {images/RaspberryPiMouse.png}
%  \caption{Example}
%  \label{Fig:Example}
% \end{figure}

% \subsubsection{etc...}
\newpage
