\documentclass[uplatex, a4paper, 12pt, openany, oneside]{jsbook}

\usepackage[dvipdfmx]{graphicx}
\usepackage[dvipdfmx]{color}
\usepackage[dvipdfmx, bookmarks=true, setpagesize=false]{hyperref}
\usepackage{pxjahyper}

\usepackage{thesis}
\usepackage{here}
\usepackage{url}


\thesis{学士論文}
\title{
  \centering
    \scalebox{1.0}{モデル予測制御を利用したヒューマノイドロボットの}\\
    \vspace{-0.3zh}
    \scalebox{1.0}{歩行パターン生成器の実装}
    \vspace{-0.6zh}
}
\setlength{\textwidth}{\fullwidth}
\setlength{\evensidemargin}{\oddsidemargin}

\date{\today}
\vspace{-15.0zh}
\teacher{林原 靖男 教授}
\vspace{-15.0zh}
\organization{千葉工業大学 先進工学部 未来ロボティクス学科}
\author{19C1012 井上叡}
\vspace{-15zh}

\renewcommand{\baselinestretch}{1.2}
\begin{document}

%% Front Matter
\frontmatter{}
%
\maketitle
%
%!TEX root = ../thesis.tex
\chapter*{概要}
\thispagestyle{empty}
%
\begin{center}
  \scalebox{1.5}{タイトル}\\
\end{center}
\vspace{1.0zh}
%


キーワード: モデル予測制御,歩行パターン生成,ヒューマノイドロボット
%
\newpage
%%
\chapter*{abstract}
\thispagestyle{empty}
%
\begin{center}
  \scalebox{1.3}{title}
\end{center}
\vspace{1.0zh}
%


keywords: Model Predictive Control,Walkingpattern Generation, Humanoid Robot 

%
\tableofcontents
%
\listoffigures
%
\listoftables
%

%
%% Main Matter
\mainmatter{}
%
\chapter{序論}
\label{chap:introduction}
%
%\input{introduction/preface}
%
%!TEX root = ../thesis.tex

\section{背景}
\section{目的}
\section{論文構成}

% \begin{figure}[hbtp]
%   \centering
%  \includegraphics[keepaspectratio, scale=0.8]
%       {images/RaspberryPiMouse.png}
%  \caption{Example}
%  \label{Fig:Example}
% \end{figure}

\subsubsection{etc...}
\newpage

%

%ここにディレクトリのパスを追加していく
%
\chapter{モデル予測制御を用いた歩行パターン生成法}
\section{ヒューマノイドロボットの歩行パターン生成法}
\subsection{ZMPをベースとした手法}
\subsection{capture pointをベースとした手法}
\section{モデル予測制御を応用した歩行パターン生成法}
\subsection{理論の歴史的経緯(別の言い方に変える)}
\section{既存の実装例}

\chapter{理論の実装}
\section{論文1}
\section{論文2}
\section{論文3}

\chapter{シミュレータによる検証}
\section{実装1}
\section{実装2}
\section{実装3}


\chapter{結論}

%% Back Matter
\backmatter{}
%
%!TEX root = ../thesis.tex
%\bibliographystyle{plain}
\bibliographystyle{junsrt}
%\bibliography{report}
\nocite{*}
\bibliography{main_bibliography}
%
%!TEX root = ../thesis.tex
\chapter*{付録}
\addcontentsline{toc}{chapter}{付録}

%
%!TEX root = ../thesis.tex
\chapter*{謝辞}
\addcontentsline{toc}{chapter}{謝辞}

本研究を進めるにあたり,1年に渡り, 熱心にご指導を頂いた林原靖男教授に深く感謝いたします.
%


%

\end{document}
