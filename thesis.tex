\documentclass[uplatex, a4paper, 12pt, openany, oneside]{jsbook}

\usepackage[dvipdfmx]{graphicx}
\usepackage[dvipdfmx]{color}
\usepackage[dvipdfmx, bookmarks=true, setpagesize=false]{hyperref}
\usepackage{pxjahyper}

\usepackage{thesis}
\usepackage{here}
\usepackage{url}


\thesis{学士論文}
\title{
  \centering
    \scalebox{1.0}{モデル予測制御を利用したヒューマノイドロボットの}\\
    \vspace{-0.3zh}
    \scalebox{1.0}{歩行パターン生成器の実装}
    \vspace{-0.6zh}
}
\setlength{\textwidth}{\fullwidth}
\setlength{\evensidemargin}{\oddsidemargin}

\date{\today}
\vspace{-15.0zh}
\teacher{林原 靖男 教授}
\vspace{-15.0zh}
\organization{千葉工業大学 先進工学部 未来ロボティクス学科}
\author{19C1012 井上叡}
\vspace{-15zh}

\renewcommand{\baselinestretch}{1.2}
\begin{document}

%% Front Matter
\frontmatter{}
%
%!TEX root = ../thesis.tex

\chapter{MPCによる歩行パターン生成}
\label{chap:mpc_walking}
%
% \input{introduction/preface}
%
%!TEX root = ../thesis.tex
\section{緒言}
本章では,二足歩行ロボットの歩行パターン生成に関する基本的な概念,及び提案手法において実装したアルゴリズムに関して述べる.

% \section{二足歩行ロボットの歩行パターン生成}
% ここでは,二足歩行ロボットの歩行パターン生成を行う際に重要となる,ロボットのモデリング方法,安定性判別手法について述べる.

\section{ZMP}
二足ロボットの安定性判別の指標として代表的なものとして,Vukobratovićらによって提案されたZero Moment Point(以下,ZMP)\cite{VUKOBRATOVIC19721}がある.ZMPはロボットが床から受ける床反力の圧力中心と定義される.以下の図\ref{Fig:zmpillust}では,ロボットの足に作用する床反力の合計と等価な力であるRの位置がZMPの位置となる.

\begin{figure}[hbtp]
  \centering
  \includegraphics[keepaspectratio, scale=0.6]
  {images/zmp_ilust.png}
  \caption{Illustration of ZMP }
  \label{Fig:zmpillust}
\end{figure}
\newpage

\section{テーブル台車モデル}
二足歩行ロボットのモデル化手法として,図\ref{Fig:tablecart}に示すようなテーブル・台車モデルがある.このモデル化により,式\eqref{eq:table-cart}のダイナミクスを導ける.ここで,$x$,$\ddot{x}$はロボットの重心位置と重心加速度,$h_{CoM}$は鉛直方向の重心高さ,$g$は重心の加速度,$p$はzmpの位置である.
式\eqref{eq:table-cart}によるモデル化に対し,システムへの入力を重心の躍度(jerk)とすると,二足歩行ロボットは以下の式\eqref{eq:teble-cart-system-state},式\eqref{eq:teble-cart-system-output}からなるシステムとして表現できる.

\begin{equation}
  p = x - \frac{h_{CoM}}{g}\ddot{x}
  \label{eq:table-cart}
\end{equation}


\begin{equation}
  \frac{d}{dt}\begin{pmatrix}
    x \\ \dot{x} \\ \ddot{x}
  \end{pmatrix}
  =
  \begin{pmatrix}
    0 & 1 & 0 \\ 0 & 0 & 1 \\ 0 & 0 & 0
  \end{pmatrix}
  \begin{pmatrix}
    x \\ \dot{x} \\ \ddot{x}
  \end{pmatrix}
  +
  \begin{pmatrix}
    0 \\ 0 \\ 1
  \end{pmatrix}
  u
  \label{eq:teble-cart-system-state}
\end{equation}


\begin{equation}
  p = \begin{pmatrix}
    1 & 0 & -\frac{h_{CoM}}{g}
  \end{pmatrix}
  \begin{pmatrix}
    x \\ \dot{x} \\ \ddot{x}
  \end{pmatrix}
  \label{eq:teble-cart-system-output}
\end{equation}

\newpage

ここで,上記のシステムを離散化すると,システムは式\eqref{eq:table-cart-quantize-variable}\textasciitilde\eqref{eq:table-cart-quantize-output}で表すことができる.


\begin{equation}
  \hat{x}_{k}=
  \begin{pmatrix}
    x(kT)\cr \dot{x}(kT)\cr\ddot{x}(kT)
  \end{pmatrix}
  ,
  % {\buildrel\ldots\over{u}}_{k}
  u_{k}
    =
    {\buildrel\ldots\over{x}}(kT)
  , z_{k}=z(kT),
  \label{eq:table-cart-quantize-variable}
\end{equation}

\begin{equation}
  \hat{x}_{k+1}=
  \begin{pmatrix}
    1 & T & T^{2}/2\cr 0 & 1 & T\cr 0 & 0 & 1
  \end{pmatrix}
  \hat{x}_{k} +
  \begin{pmatrix}
    T^{3}/6 \\  T^{3}/2 \\ T
  \end{pmatrix}
  u_{k}
  \label{eq:table-cart-quantize-state}
\end{equation}

\begin{equation}
  z_{k} =
  \begin{pmatrix}
    1 & 0 & \frac{h_{CoM}}{g}
  \end{pmatrix}
  \hat{x}_{k}
  \label{eq:table-cart-quantize-output}
\end{equation}

\begin{figure}[hbtp]
  \centering
  \includegraphics[keepaspectratio, scale=0.6]
  {images/table_cart_hcom.png}
  \caption{Illustration of table-cart model }
  \label{Fig:tablecart}
\end{figure}



\newpage

\section{予見制御による歩行パターン生成手法}
上記のZMPとテーブル台車モデルの概念を利用した歩行パターン生成手法としては,kajitaらによる予見制御を利用した歩行パターン生成手法\cite{PREVIEW}がある.kajitaらによる手法では\eqref{eq:table-cart-quantize-variable}\textasciitilde\eqref{eq:table-cart-quantize-output}で表現したロボットのダイナミクスに対して,予見制御理論によって以下の評価関数式\eqref{eq:preview_valuefunc}を最小化する制御入力(二足歩行ロボットの歩行パターン)を得る.

\begin{equation}
  \min_{{\buildrel\ldots\over{x}}_{k},{\buildrel\ldots\over{x}}_{k+1},\ldots}\sum_{i=k}^{\infty}{1\over 2}Q\left(z_{i+1}-z_{i+1}^{ref}\right)^{2}+{1\over 2}R{\buildrel\ldots\over{x}}_{i}^{2}
  \label{eq:preview_valuefunc}
\end{equation}

式\eqref{eq:preview_valuefunc}内,$z$はZMPの位置,$z^{ref}$はZMPの目標位置,$u$は入力である重心の躍度(jerk),$Q$と$R$は重みを示す.

\section{MPCを利用した歩行パターン生成手法}
本研究では,kajitaらによる予見制御を利用した歩行パターン生成手法\cite{PREVIEW}をMPCによって拡張した,Wieberらによる手法\cite{WIEBER}を実装する.ここでは,その手法\cite{WIEBER}について紹介する.

まず,MPCは最適制御の一種であり,無限長の区間に渡る最適化問題を解く代わりに,有限長の区間で最適化問題を解いて最適入力を得る手法である.有限区間の最適化問題であるがゆえに,陽に表現した制約を考慮しながら最適化問題を解くことができるのが特徴である.また,有限区間での最適化問題を解く際には,有限区間に渡る未来の状態,ないしは出力の予測値を利用して最適化を行う.
式\eqref{eq:control-general-system}で表されるダイナミクスからなる一般的なシステムでは,状態の予測は式\eqref{eq:mpc-preview}で表される.

\begin{equation}
  \begin{cases}
    \dot{x} & = Ax + Bu \\
    y       & = Cx
  \end{cases}
  \label{eq:control-general-system}
\end{equation}

\begin{equation}
  \begin{pmatrix}
    x_{k+1} \\x_{k+2}\\x_{k+3}\\
    \vdots  \\ x_{k+n}
  \end{pmatrix}
  =
  \begin{pmatrix}
    A      \\A^2\\A^3\\
    \vdots \\ A^n
  \end{pmatrix}
  x_{k} +
  \begin{pmatrix}
    B        & 0        & \cdots &        & 0 \\
    AB       & B        & 0      & \cdots & 0 \\
    A^2B     & AB       & B      & \cdots & 0 \\
    \vdots   & \vdots   &        &        & 0 \\
    A^{n-1}B & A^{n-2}B & \cdots & AB     & B
  \end{pmatrix}
  \begin{pmatrix}
    u_{k}   \\
    u_{k+1} \\
    u_{k+2} \\
    \vdots  \\
    u_{k+n-1}
  \end{pmatrix}
  \label{eq:mpc-preview}
\end{equation}

また,式\eqref{eq:mpc-preview}に対して,式\eqref{eq:mpc-preview-c}を掛けることで出力の予測は式\eqref{eq:mpc-preview-output}と表せる.

\begin{equation}
  \mathcal{C} =
  \begin{pmatrix}
    C      & 0 & \cdots & 0 \\
    0      & C & \cdots & 0 \\
    \vdots &   &    \ddots  & 0 \\
    0 & \cdots & & C\\
  \end{pmatrix}
  \label{eq:mpc-preview-c}
\end{equation}

\begin{equation}
  \begin{pmatrix}
    z_{k+1} \\z_{k+2}\\z_{k+3}\\
    \vdots  \\ z_{k+n}
  \end{pmatrix}
  =
  \begin{pmatrix}
    C      & 0 & \cdots & 0 \\
    0      & C & \cdots & 0 \\
    \vdots &   &    \ddots  & 0 \\
    0 & \cdots & & C\\
  \end{pmatrix}
  \begin{pmatrix}
    x_{k+1} \\x_{k+2}\\x_{k+3}\\
    \vdots  \\ x_{k+n}
  \end{pmatrix}
  \label{eq:mpc-preview-output}
\end{equation}

Wieberらによる手法\cite{WIEBER}では,式\eqref{eq:table-cart-quantize-variable}\textasciitilde\eqref{eq:table-cart-quantize-output}からなるダイナミクスを元に,式\eqref{eq:mpc_valuefunc}を評価関数とする.

\begin{equation}
  \min_{{\buildrel\ldots\over{x}}_{k},{\buildrel\ldots\over{x}}_{k+1},\ldots,{\buildrel\ldots\over{x}}_{k+n}}\sum_{i=k}^{n}{1\over 2}Q\left(z_{i+1}-z_{i+1}^{ref}\right)^{2}+{1\over 2}R{\buildrel\ldots\over{x}}_{i}^{2}
  \label{eq:mpc_valuefunc}
\end{equation}

ここで,$z^{ref}$ は目標zmp(目標出力),$z$は現在のzmp(現在の出力)$\dddot{x}$は重心のjerk(入力)となる.
この評価関数とkajitaらによる手法の式\eqref{eq:preview_valuefunc}の評価関数の違いは評価区間である.
kajitaらの式では評価区間が無限区間であるのに対して,MPCを利用したwieberらの式は評価区間の長さがNとなっている.
上記の評価関数に対して,式\eqref{eq:inequality-constraint-input},式\eqref{eq:inequality-constraint-state}で表される不等式制約を与える.

\begin{equation}
  \begin{pmatrix}
    Lu_{0} \\ \vdots \\ Lu_{n}
  \end{pmatrix}
  \leq \dddot{x} \leq
  \begin{pmatrix}
    Uu_{0} \\ \vdots \\ Uu_{n}
  \end{pmatrix}
  \label{eq:inequality-constraint-input}
\end{equation}

\begin{equation}
  \begin{pmatrix}
    Lx_{0} \\ \vdots \\ Lx_{n}
  \end{pmatrix}
  \leq z \leq
  \begin{pmatrix}
    Ux_{0} \\ \vdots \\ Ux_{n}
  \end{pmatrix}
  \label{eq:inequality-constraint-state}
\end{equation}

% \begin{equation}
%   \dddot{x} =
%   \begin{pmatrix}
%     e_{0} \\ \vdots \\ e_{n}
%   \end{pmatrix}
%   \label{eq:equality-constraint}
% \end{equation}

上記の不等式制約では,入力と出力に対して,上下限が設定される.入力の上下限はモーターの最大速度等,ロボットの特性によって決定される.出力制約は支持多角形内でのZMPの位置に対する制約となる.ZMPが支持多角形の境界部にある時に,ロボットは転倒を始めるので,支持多角形の境界部から一定の距離を取った範囲を制約の範囲とする.

評価関数式\eqref{eq:mpc_valuefunc}を二次計画法の問題として捉え,式\eqref{eq:inequality-constraint-input}と式\eqref{eq:inequality-constraint-state}からなる制約を考慮して制御ステップ毎に解くことでそのステップにおける最適入力を得る.
%

%
%% Main Matter
\mainmatter{}
%
%!TEX root = ../thesis.tex

\chapter{MPCによる歩行パターン生成}
\label{chap:mpc_walking}
%
% \input{introduction/preface}
%
%!TEX root = ../thesis.tex
\section{緒言}
本章では,二足歩行ロボットの歩行パターン生成に関する基本的な概念,及び提案手法において実装したアルゴリズムに関して述べる.

% \section{二足歩行ロボットの歩行パターン生成}
% ここでは,二足歩行ロボットの歩行パターン生成を行う際に重要となる,ロボットのモデリング方法,安定性判別手法について述べる.

\section{ZMP}
二足ロボットの安定性判別の指標として代表的なものとして,Vukobratovićらによって提案されたZero Moment Point(以下,ZMP)\cite{VUKOBRATOVIC19721}がある.ZMPはロボットが床から受ける床反力の圧力中心と定義される.以下の図\ref{Fig:zmpillust}では,ロボットの足に作用する床反力の合計と等価な力であるRの位置がZMPの位置となる.

\begin{figure}[hbtp]
  \centering
  \includegraphics[keepaspectratio, scale=0.6]
  {images/zmp_ilust.png}
  \caption{Illustration of ZMP }
  \label{Fig:zmpillust}
\end{figure}
\newpage

\section{テーブル台車モデル}
二足歩行ロボットのモデル化手法として,図\ref{Fig:tablecart}に示すようなテーブル・台車モデルがある.このモデル化により,式\eqref{eq:table-cart}のダイナミクスを導ける.ここで,$x$,$\ddot{x}$はロボットの重心位置と重心加速度,$h_{CoM}$は鉛直方向の重心高さ,$g$は重心の加速度,$p$はzmpの位置である.
式\eqref{eq:table-cart}によるモデル化に対し,システムへの入力を重心の躍度(jerk)とすると,二足歩行ロボットは以下の式\eqref{eq:teble-cart-system-state},式\eqref{eq:teble-cart-system-output}からなるシステムとして表現できる.

\begin{equation}
  p = x - \frac{h_{CoM}}{g}\ddot{x}
  \label{eq:table-cart}
\end{equation}


\begin{equation}
  \frac{d}{dt}\begin{pmatrix}
    x \\ \dot{x} \\ \ddot{x}
  \end{pmatrix}
  =
  \begin{pmatrix}
    0 & 1 & 0 \\ 0 & 0 & 1 \\ 0 & 0 & 0
  \end{pmatrix}
  \begin{pmatrix}
    x \\ \dot{x} \\ \ddot{x}
  \end{pmatrix}
  +
  \begin{pmatrix}
    0 \\ 0 \\ 1
  \end{pmatrix}
  u
  \label{eq:teble-cart-system-state}
\end{equation}


\begin{equation}
  p = \begin{pmatrix}
    1 & 0 & -\frac{h_{CoM}}{g}
  \end{pmatrix}
  \begin{pmatrix}
    x \\ \dot{x} \\ \ddot{x}
  \end{pmatrix}
  \label{eq:teble-cart-system-output}
\end{equation}

\newpage

ここで,上記のシステムを離散化すると,システムは式\eqref{eq:table-cart-quantize-variable}\textasciitilde\eqref{eq:table-cart-quantize-output}で表すことができる.


\begin{equation}
  \hat{x}_{k}=
  \begin{pmatrix}
    x(kT)\cr \dot{x}(kT)\cr\ddot{x}(kT)
  \end{pmatrix}
  ,
  % {\buildrel\ldots\over{u}}_{k}
  u_{k}
    =
    {\buildrel\ldots\over{x}}(kT)
  , z_{k}=z(kT),
  \label{eq:table-cart-quantize-variable}
\end{equation}

\begin{equation}
  \hat{x}_{k+1}=
  \begin{pmatrix}
    1 & T & T^{2}/2\cr 0 & 1 & T\cr 0 & 0 & 1
  \end{pmatrix}
  \hat{x}_{k} +
  \begin{pmatrix}
    T^{3}/6 \\  T^{3}/2 \\ T
  \end{pmatrix}
  u_{k}
  \label{eq:table-cart-quantize-state}
\end{equation}

\begin{equation}
  z_{k} =
  \begin{pmatrix}
    1 & 0 & \frac{h_{CoM}}{g}
  \end{pmatrix}
  \hat{x}_{k}
  \label{eq:table-cart-quantize-output}
\end{equation}

\begin{figure}[hbtp]
  \centering
  \includegraphics[keepaspectratio, scale=0.6]
  {images/table_cart_hcom.png}
  \caption{Illustration of table-cart model }
  \label{Fig:tablecart}
\end{figure}



\newpage

\section{予見制御による歩行パターン生成手法}
上記のZMPとテーブル台車モデルの概念を利用した歩行パターン生成手法としては,kajitaらによる予見制御を利用した歩行パターン生成手法\cite{PREVIEW}がある.kajitaらによる手法では\eqref{eq:table-cart-quantize-variable}\textasciitilde\eqref{eq:table-cart-quantize-output}で表現したロボットのダイナミクスに対して,予見制御理論によって以下の評価関数式\eqref{eq:preview_valuefunc}を最小化する制御入力(二足歩行ロボットの歩行パターン)を得る.

\begin{equation}
  \min_{{\buildrel\ldots\over{x}}_{k},{\buildrel\ldots\over{x}}_{k+1},\ldots}\sum_{i=k}^{\infty}{1\over 2}Q\left(z_{i+1}-z_{i+1}^{ref}\right)^{2}+{1\over 2}R{\buildrel\ldots\over{x}}_{i}^{2}
  \label{eq:preview_valuefunc}
\end{equation}

式\eqref{eq:preview_valuefunc}内,$z$はZMPの位置,$z^{ref}$はZMPの目標位置,$u$は入力である重心の躍度(jerk),$Q$と$R$は重みを示す.

\section{MPCを利用した歩行パターン生成手法}
本研究では,kajitaらによる予見制御を利用した歩行パターン生成手法\cite{PREVIEW}をMPCによって拡張した,Wieberらによる手法\cite{WIEBER}を実装する.ここでは,その手法\cite{WIEBER}について紹介する.

まず,MPCは最適制御の一種であり,無限長の区間に渡る最適化問題を解く代わりに,有限長の区間で最適化問題を解いて最適入力を得る手法である.有限区間の最適化問題であるがゆえに,陽に表現した制約を考慮しながら最適化問題を解くことができるのが特徴である.また,有限区間での最適化問題を解く際には,有限区間に渡る未来の状態,ないしは出力の予測値を利用して最適化を行う.
式\eqref{eq:control-general-system}で表されるダイナミクスからなる一般的なシステムでは,状態の予測は式\eqref{eq:mpc-preview}で表される.

\begin{equation}
  \begin{cases}
    \dot{x} & = Ax + Bu \\
    y       & = Cx
  \end{cases}
  \label{eq:control-general-system}
\end{equation}

\begin{equation}
  \begin{pmatrix}
    x_{k+1} \\x_{k+2}\\x_{k+3}\\
    \vdots  \\ x_{k+n}
  \end{pmatrix}
  =
  \begin{pmatrix}
    A      \\A^2\\A^3\\
    \vdots \\ A^n
  \end{pmatrix}
  x_{k} +
  \begin{pmatrix}
    B        & 0        & \cdots &        & 0 \\
    AB       & B        & 0      & \cdots & 0 \\
    A^2B     & AB       & B      & \cdots & 0 \\
    \vdots   & \vdots   &        &        & 0 \\
    A^{n-1}B & A^{n-2}B & \cdots & AB     & B
  \end{pmatrix}
  \begin{pmatrix}
    u_{k}   \\
    u_{k+1} \\
    u_{k+2} \\
    \vdots  \\
    u_{k+n-1}
  \end{pmatrix}
  \label{eq:mpc-preview}
\end{equation}

また,式\eqref{eq:mpc-preview}に対して,式\eqref{eq:mpc-preview-c}を掛けることで出力の予測は式\eqref{eq:mpc-preview-output}と表せる.

\begin{equation}
  \mathcal{C} =
  \begin{pmatrix}
    C      & 0 & \cdots & 0 \\
    0      & C & \cdots & 0 \\
    \vdots &   &    \ddots  & 0 \\
    0 & \cdots & & C\\
  \end{pmatrix}
  \label{eq:mpc-preview-c}
\end{equation}

\begin{equation}
  \begin{pmatrix}
    z_{k+1} \\z_{k+2}\\z_{k+3}\\
    \vdots  \\ z_{k+n}
  \end{pmatrix}
  =
  \begin{pmatrix}
    C      & 0 & \cdots & 0 \\
    0      & C & \cdots & 0 \\
    \vdots &   &    \ddots  & 0 \\
    0 & \cdots & & C\\
  \end{pmatrix}
  \begin{pmatrix}
    x_{k+1} \\x_{k+2}\\x_{k+3}\\
    \vdots  \\ x_{k+n}
  \end{pmatrix}
  \label{eq:mpc-preview-output}
\end{equation}

Wieberらによる手法\cite{WIEBER}では,式\eqref{eq:table-cart-quantize-variable}\textasciitilde\eqref{eq:table-cart-quantize-output}からなるダイナミクスを元に,式\eqref{eq:mpc_valuefunc}を評価関数とする.

\begin{equation}
  \min_{{\buildrel\ldots\over{x}}_{k},{\buildrel\ldots\over{x}}_{k+1},\ldots,{\buildrel\ldots\over{x}}_{k+n}}\sum_{i=k}^{n}{1\over 2}Q\left(z_{i+1}-z_{i+1}^{ref}\right)^{2}+{1\over 2}R{\buildrel\ldots\over{x}}_{i}^{2}
  \label{eq:mpc_valuefunc}
\end{equation}

ここで,$z^{ref}$ は目標zmp(目標出力),$z$は現在のzmp(現在の出力)$\dddot{x}$は重心のjerk(入力)となる.
この評価関数とkajitaらによる手法の式\eqref{eq:preview_valuefunc}の評価関数の違いは評価区間である.
kajitaらの式では評価区間が無限区間であるのに対して,MPCを利用したwieberらの式は評価区間の長さがNとなっている.
上記の評価関数に対して,式\eqref{eq:inequality-constraint-input},式\eqref{eq:inequality-constraint-state}で表される不等式制約を与える.

\begin{equation}
  \begin{pmatrix}
    Lu_{0} \\ \vdots \\ Lu_{n}
  \end{pmatrix}
  \leq \dddot{x} \leq
  \begin{pmatrix}
    Uu_{0} \\ \vdots \\ Uu_{n}
  \end{pmatrix}
  \label{eq:inequality-constraint-input}
\end{equation}

\begin{equation}
  \begin{pmatrix}
    Lx_{0} \\ \vdots \\ Lx_{n}
  \end{pmatrix}
  \leq z \leq
  \begin{pmatrix}
    Ux_{0} \\ \vdots \\ Ux_{n}
  \end{pmatrix}
  \label{eq:inequality-constraint-state}
\end{equation}

% \begin{equation}
%   \dddot{x} =
%   \begin{pmatrix}
%     e_{0} \\ \vdots \\ e_{n}
%   \end{pmatrix}
%   \label{eq:equality-constraint}
% \end{equation}

上記の不等式制約では,入力と出力に対して,上下限が設定される.入力の上下限はモーターの最大速度等,ロボットの特性によって決定される.出力制約は支持多角形内でのZMPの位置に対する制約となる.ZMPが支持多角形の境界部にある時に,ロボットは転倒を始めるので,支持多角形の境界部から一定の距離を取った範囲を制約の範囲とする.

評価関数式\eqref{eq:mpc_valuefunc}を二次計画法の問題として捉え,式\eqref{eq:inequality-constraint-input}と式\eqref{eq:inequality-constraint-state}からなる制約を考慮して制御ステップ毎に解くことでそのステップにおける最適入力を得る.
%

%ここにディレクトリのパスを追加していく
%
\chapter{モデル予測制御を用いた歩行パターン生成法}
\section{ヒューマノイドロボットの歩行パターン生成法}
\subsection{ZMPをベースとした手法}
\subsection{capture pointをベースとした手法}
\section{モデル予測制御を応用した歩行パターン生成法}
\subsection{理論の歴史的経緯(別の言い方に変える)}
\section{既存の実装例}

\chapter{理論の実装}
\section{論文1}
\section{論文2}
\section{論文3}

\chapter{シミュレータによる検証}
\section{実装1}
\section{実装2}
\section{実装3}


\chapter{結論}

%% Back Matter
\backmatter{}
%
%!TEX root = ../thesis.tex

\chapter{MPCによる歩行パターン生成}
\label{chap:mpc_walking}
%
% \input{introduction/preface}
%
%!TEX root = ../thesis.tex
\section{緒言}
本章では,二足歩行ロボットの歩行パターン生成に関する基本的な概念,及び提案手法において実装したアルゴリズムに関して述べる.

% \section{二足歩行ロボットの歩行パターン生成}
% ここでは,二足歩行ロボットの歩行パターン生成を行う際に重要となる,ロボットのモデリング方法,安定性判別手法について述べる.

\section{ZMP}
二足ロボットの安定性判別の指標として代表的なものとして,Vukobratovićらによって提案されたZero Moment Point(以下,ZMP)\cite{VUKOBRATOVIC19721}がある.ZMPはロボットが床から受ける床反力の圧力中心と定義される.以下の図\ref{Fig:zmpillust}では,ロボットの足に作用する床反力の合計と等価な力であるRの位置がZMPの位置となる.

\begin{figure}[hbtp]
  \centering
  \includegraphics[keepaspectratio, scale=0.6]
  {images/zmp_ilust.png}
  \caption{Illustration of ZMP }
  \label{Fig:zmpillust}
\end{figure}
\newpage

\section{テーブル台車モデル}
二足歩行ロボットのモデル化手法として,図\ref{Fig:tablecart}に示すようなテーブル・台車モデルがある.このモデル化により,式\eqref{eq:table-cart}のダイナミクスを導ける.ここで,$x$,$\ddot{x}$はロボットの重心位置と重心加速度,$h_{CoM}$は鉛直方向の重心高さ,$g$は重心の加速度,$p$はzmpの位置である.
式\eqref{eq:table-cart}によるモデル化に対し,システムへの入力を重心の躍度(jerk)とすると,二足歩行ロボットは以下の式\eqref{eq:teble-cart-system-state},式\eqref{eq:teble-cart-system-output}からなるシステムとして表現できる.

\begin{equation}
  p = x - \frac{h_{CoM}}{g}\ddot{x}
  \label{eq:table-cart}
\end{equation}


\begin{equation}
  \frac{d}{dt}\begin{pmatrix}
    x \\ \dot{x} \\ \ddot{x}
  \end{pmatrix}
  =
  \begin{pmatrix}
    0 & 1 & 0 \\ 0 & 0 & 1 \\ 0 & 0 & 0
  \end{pmatrix}
  \begin{pmatrix}
    x \\ \dot{x} \\ \ddot{x}
  \end{pmatrix}
  +
  \begin{pmatrix}
    0 \\ 0 \\ 1
  \end{pmatrix}
  u
  \label{eq:teble-cart-system-state}
\end{equation}


\begin{equation}
  p = \begin{pmatrix}
    1 & 0 & -\frac{h_{CoM}}{g}
  \end{pmatrix}
  \begin{pmatrix}
    x \\ \dot{x} \\ \ddot{x}
  \end{pmatrix}
  \label{eq:teble-cart-system-output}
\end{equation}

\newpage

ここで,上記のシステムを離散化すると,システムは式\eqref{eq:table-cart-quantize-variable}\textasciitilde\eqref{eq:table-cart-quantize-output}で表すことができる.


\begin{equation}
  \hat{x}_{k}=
  \begin{pmatrix}
    x(kT)\cr \dot{x}(kT)\cr\ddot{x}(kT)
  \end{pmatrix}
  ,
  % {\buildrel\ldots\over{u}}_{k}
  u_{k}
    =
    {\buildrel\ldots\over{x}}(kT)
  , z_{k}=z(kT),
  \label{eq:table-cart-quantize-variable}
\end{equation}

\begin{equation}
  \hat{x}_{k+1}=
  \begin{pmatrix}
    1 & T & T^{2}/2\cr 0 & 1 & T\cr 0 & 0 & 1
  \end{pmatrix}
  \hat{x}_{k} +
  \begin{pmatrix}
    T^{3}/6 \\  T^{3}/2 \\ T
  \end{pmatrix}
  u_{k}
  \label{eq:table-cart-quantize-state}
\end{equation}

\begin{equation}
  z_{k} =
  \begin{pmatrix}
    1 & 0 & \frac{h_{CoM}}{g}
  \end{pmatrix}
  \hat{x}_{k}
  \label{eq:table-cart-quantize-output}
\end{equation}

\begin{figure}[hbtp]
  \centering
  \includegraphics[keepaspectratio, scale=0.6]
  {images/table_cart_hcom.png}
  \caption{Illustration of table-cart model }
  \label{Fig:tablecart}
\end{figure}



\newpage

\section{予見制御による歩行パターン生成手法}
上記のZMPとテーブル台車モデルの概念を利用した歩行パターン生成手法としては,kajitaらによる予見制御を利用した歩行パターン生成手法\cite{PREVIEW}がある.kajitaらによる手法では\eqref{eq:table-cart-quantize-variable}\textasciitilde\eqref{eq:table-cart-quantize-output}で表現したロボットのダイナミクスに対して,予見制御理論によって以下の評価関数式\eqref{eq:preview_valuefunc}を最小化する制御入力(二足歩行ロボットの歩行パターン)を得る.

\begin{equation}
  \min_{{\buildrel\ldots\over{x}}_{k},{\buildrel\ldots\over{x}}_{k+1},\ldots}\sum_{i=k}^{\infty}{1\over 2}Q\left(z_{i+1}-z_{i+1}^{ref}\right)^{2}+{1\over 2}R{\buildrel\ldots\over{x}}_{i}^{2}
  \label{eq:preview_valuefunc}
\end{equation}

式\eqref{eq:preview_valuefunc}内,$z$はZMPの位置,$z^{ref}$はZMPの目標位置,$u$は入力である重心の躍度(jerk),$Q$と$R$は重みを示す.

\section{MPCを利用した歩行パターン生成手法}
本研究では,kajitaらによる予見制御を利用した歩行パターン生成手法\cite{PREVIEW}をMPCによって拡張した,Wieberらによる手法\cite{WIEBER}を実装する.ここでは,その手法\cite{WIEBER}について紹介する.

まず,MPCは最適制御の一種であり,無限長の区間に渡る最適化問題を解く代わりに,有限長の区間で最適化問題を解いて最適入力を得る手法である.有限区間の最適化問題であるがゆえに,陽に表現した制約を考慮しながら最適化問題を解くことができるのが特徴である.また,有限区間での最適化問題を解く際には,有限区間に渡る未来の状態,ないしは出力の予測値を利用して最適化を行う.
式\eqref{eq:control-general-system}で表されるダイナミクスからなる一般的なシステムでは,状態の予測は式\eqref{eq:mpc-preview}で表される.

\begin{equation}
  \begin{cases}
    \dot{x} & = Ax + Bu \\
    y       & = Cx
  \end{cases}
  \label{eq:control-general-system}
\end{equation}

\begin{equation}
  \begin{pmatrix}
    x_{k+1} \\x_{k+2}\\x_{k+3}\\
    \vdots  \\ x_{k+n}
  \end{pmatrix}
  =
  \begin{pmatrix}
    A      \\A^2\\A^3\\
    \vdots \\ A^n
  \end{pmatrix}
  x_{k} +
  \begin{pmatrix}
    B        & 0        & \cdots &        & 0 \\
    AB       & B        & 0      & \cdots & 0 \\
    A^2B     & AB       & B      & \cdots & 0 \\
    \vdots   & \vdots   &        &        & 0 \\
    A^{n-1}B & A^{n-2}B & \cdots & AB     & B
  \end{pmatrix}
  \begin{pmatrix}
    u_{k}   \\
    u_{k+1} \\
    u_{k+2} \\
    \vdots  \\
    u_{k+n-1}
  \end{pmatrix}
  \label{eq:mpc-preview}
\end{equation}

また,式\eqref{eq:mpc-preview}に対して,式\eqref{eq:mpc-preview-c}を掛けることで出力の予測は式\eqref{eq:mpc-preview-output}と表せる.

\begin{equation}
  \mathcal{C} =
  \begin{pmatrix}
    C      & 0 & \cdots & 0 \\
    0      & C & \cdots & 0 \\
    \vdots &   &    \ddots  & 0 \\
    0 & \cdots & & C\\
  \end{pmatrix}
  \label{eq:mpc-preview-c}
\end{equation}

\begin{equation}
  \begin{pmatrix}
    z_{k+1} \\z_{k+2}\\z_{k+3}\\
    \vdots  \\ z_{k+n}
  \end{pmatrix}
  =
  \begin{pmatrix}
    C      & 0 & \cdots & 0 \\
    0      & C & \cdots & 0 \\
    \vdots &   &    \ddots  & 0 \\
    0 & \cdots & & C\\
  \end{pmatrix}
  \begin{pmatrix}
    x_{k+1} \\x_{k+2}\\x_{k+3}\\
    \vdots  \\ x_{k+n}
  \end{pmatrix}
  \label{eq:mpc-preview-output}
\end{equation}

Wieberらによる手法\cite{WIEBER}では,式\eqref{eq:table-cart-quantize-variable}\textasciitilde\eqref{eq:table-cart-quantize-output}からなるダイナミクスを元に,式\eqref{eq:mpc_valuefunc}を評価関数とする.

\begin{equation}
  \min_{{\buildrel\ldots\over{x}}_{k},{\buildrel\ldots\over{x}}_{k+1},\ldots,{\buildrel\ldots\over{x}}_{k+n}}\sum_{i=k}^{n}{1\over 2}Q\left(z_{i+1}-z_{i+1}^{ref}\right)^{2}+{1\over 2}R{\buildrel\ldots\over{x}}_{i}^{2}
  \label{eq:mpc_valuefunc}
\end{equation}

ここで,$z^{ref}$ は目標zmp(目標出力),$z$は現在のzmp(現在の出力)$\dddot{x}$は重心のjerk(入力)となる.
この評価関数とkajitaらによる手法の式\eqref{eq:preview_valuefunc}の評価関数の違いは評価区間である.
kajitaらの式では評価区間が無限区間であるのに対して,MPCを利用したwieberらの式は評価区間の長さがNとなっている.
上記の評価関数に対して,式\eqref{eq:inequality-constraint-input},式\eqref{eq:inequality-constraint-state}で表される不等式制約を与える.

\begin{equation}
  \begin{pmatrix}
    Lu_{0} \\ \vdots \\ Lu_{n}
  \end{pmatrix}
  \leq \dddot{x} \leq
  \begin{pmatrix}
    Uu_{0} \\ \vdots \\ Uu_{n}
  \end{pmatrix}
  \label{eq:inequality-constraint-input}
\end{equation}

\begin{equation}
  \begin{pmatrix}
    Lx_{0} \\ \vdots \\ Lx_{n}
  \end{pmatrix}
  \leq z \leq
  \begin{pmatrix}
    Ux_{0} \\ \vdots \\ Ux_{n}
  \end{pmatrix}
  \label{eq:inequality-constraint-state}
\end{equation}

% \begin{equation}
%   \dddot{x} =
%   \begin{pmatrix}
%     e_{0} \\ \vdots \\ e_{n}
%   \end{pmatrix}
%   \label{eq:equality-constraint}
% \end{equation}

上記の不等式制約では,入力と出力に対して,上下限が設定される.入力の上下限はモーターの最大速度等,ロボットの特性によって決定される.出力制約は支持多角形内でのZMPの位置に対する制約となる.ZMPが支持多角形の境界部にある時に,ロボットは転倒を始めるので,支持多角形の境界部から一定の距離を取った範囲を制約の範囲とする.

評価関数式\eqref{eq:mpc_valuefunc}を二次計画法の問題として捉え,式\eqref{eq:inequality-constraint-input}と式\eqref{eq:inequality-constraint-state}からなる制約を考慮して制御ステップ毎に解くことでそのステップにおける最適入力を得る.
%

%

\end{document}
