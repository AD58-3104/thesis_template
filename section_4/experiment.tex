%!TEX root = ../thesis.tex
\section{緒言}
本章では3章で実装した歩行パターン生成器を用いて,実際に歩行パターンを生成した結果を示す.
\section{実験の設定}
本研究で開発したソフトウェアを使用し,表\ref{tb:parametor}に示すパラメータで重心とZMPの軌道を3000[ms]の長さに渡って生成した.

\begin{table}[htbp]
  \centering
  \begin{tabular}{|c|c|} \hline
    Control Horizon & 1.5 [$s$] \\ \hline
    Unit time &10 [$ms$] \\ \hline
    Q / R  & 100000.0 \\ \hline
    Height of CoM & 0.6 [$m$] \\ \hline
    Step width & 0.15 [$m$] \\ \hline
    Upper bound of difference between ref and current output. & 0.02 [$m$] \\ \hline
    Lower bound of difference between ref and current output. & -0.02 [$m$] \\ \hline
    Upper bound of input & 100 [$m/s^{3}$] \\ \hline
    Lower bound of input & -100 [$m/s^{3}$] \\ \hline
  \end{tabular}
  \caption{Libraries used for implementation}
  \label{tb:parametor}
\end{table}

\newpage

次に,生成した重心とZMPの軌道,最適入力を図\ref{Fig:zmptrajectory}と図\ref{Fig:optimalinput}に示す.

\begin{figure}[hbtp]
  \centering
 \includegraphics[keepaspectratio, scale=0.6]
      {images/zmp_trajectory.png}
 \caption{Generated zmp and CoM trajectory }
 \label{Fig:zmptrajectory}
\end{figure}

\begin{figure}[hbtp]
  \centering
 \includegraphics[keepaspectratio, scale=0.6]
      {images/calculated_input.png}
\caption{Optimal input}
 \label{Fig:optimalinput}
\end{figure}

\newpage
図\ref{Fig:zmptrajectory} より,ZMP が目標 ZMP に追従する軌道が生成できていることが分かる. また,図\ref{Fig:zmptrajectory}より状態が ,図\ref{Fig:optimalinput}より入力が設定した制約の範囲内に留まっており,適切な歩行パターンが生成できていることも分かる.


