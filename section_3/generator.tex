%!TEX root = ../thesis.tex
\section{緒言}
本章では、本研究での歩行パターン生成器の実装手法、及び公開した情報について述べる。

\section{公開する情報}
本研究では、公開する内容として、文献等で明文化されにくい部分を特に対象とする。現状、文献等での明文化が少ないと筆者が捉えているのは以下の3種の情報である。
\begin{enumerate}
  \item 具体的なソースコードの実装
  \item 実際に歩行パターンを生成可能なパラメータの例
  \item 実装時に得た知見
\end{enumerate}

以上3種の情報に対して、本研究では具体的に以下に示す情報を公開する。

\begin{itemize}
  \item C++言語で実装したMPCによる歩行パターン生成器のソースコード
  \item 実装した歩行パターン生成器で歩行パターンを生成可能なパラメータの一例
  \item 式()で示されるMPCの標準形である評価関数を最適化ソルバーが求める形式へ変形する手法
  \item 上記の実装を行う上で、筆者が得た知見
\end{itemize}

上記に対して、インターネット上で誰でもアクセスできるGithub\cite{MYGITHUB}にて情報の公開を行う。

\section{実装手法}

本研究では、以下の表に示すライブラリ群を利用し、C++言語にて実装を行った。

\begin{table}[htbp]
  \centering
  \begin{tabular}{|c|c|} \hline
    QP solver & osqp\cite{OSQP} \\ \hline
    C++ binding of osqp &  osqp-eigen\cite{OSQPEIGEN} \\ \hline
    Linear algebra library &  eigen-3.4.0\cite{EIGEN} \\ \hline
  \end{tabular}
  \caption{Libraries used for implementation}
  \label{tb:mulcol}
\end{table}


\subsection{}
